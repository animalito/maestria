\documentclass[ignorenonframetext,]{beamer}
\usetheme{Berkeley}
\usecolortheme{beetle}
\usepackage{amssymb,amsmath}
\usepackage{ifxetex,ifluatex}
\usepackage{fixltx2e} % provides \textsubscript
\usepackage{lmodern}
\ifxetex
  \usepackage{fontspec,xltxtra,xunicode}
  \defaultfontfeatures{Mapping=tex-text,Scale=MatchLowercase}
  \newcommand{\euro}{€}
\else
  \ifluatex
    \usepackage{fontspec}
    \defaultfontfeatures{Mapping=tex-text,Scale=MatchLowercase}
    \newcommand{\euro}{€}
  \else
    \usepackage[T1]{fontenc}
    \usepackage[utf8]{inputenc}
      \fi
\fi
\IfFileExists{upquote.sty}{\usepackage{upquote}}{}
% use microtype if available
\IfFileExists{microtype.sty}{\usepackage{microtype}}{}

% Comment these out if you don't want a slide with just the
% part/section/subsection/subsubsection title:
\AtBeginPart{
  \let\insertpartnumber\relax
  \let\partname\relax
  \frame{\partpage}
}
\AtBeginSection{
  \let\insertsectionnumber\relax
  \let\sectionname\relax
  \frame{\sectionpage}
}
\AtBeginSubsection{
  \let\insertsubsectionnumber\relax
  \let\subsectionname\relax
  \frame{\subsectionpage}
}

\setlength{\parindent}{0pt}
\setlength{\parskip}{6pt plus 2pt minus 1pt}
\setlength{\emergencystretch}{3em}  % prevent overfull lines
\setcounter{secnumdepth}{0}

\title{Anteproyecto GLM}
\author{Carlos Petricioli, Andrea Fernandez, Andrea Garcia}
\date{09/11/2014}

\begin{document}
\frame{\titlepage}

\begin{frame}{Introducción}

\begin{itemize}
\itemsep1pt\parskip0pt\parsep0pt
\item
  Nos interesa modelar la violencia y el delito en el territorio
  mexicano enfocándonos en las zonas definidas como \emph{prioritarias}
  y teniendo como base los factores de \emph{riesgo identificados} como
  precursores de la violencia y el delito.
\end{itemize}

\end{frame}

\begin{frame}{Planteamiento del problema}

\begin{itemize}
\itemsep1pt\parskip0pt\parsep0pt
\item
  A partir de la creación del Programa Nacional para la Prevención
  Social de la Violencia y la Delincuencia 2014-2018, se ha creado la
  necesidad de tener un conjunto ordenado de indicadores que permita dar
  seguimiento, evaluar y generar las recomendaciones necesarias para que
  año a año se cumpla el objeto de atender los factores de riesgo y de
  protección vinculados a la violencia y a la delincuencia.
\end{itemize}

\end{frame}

\begin{frame}{Planteamiento del problema}

\begin{itemize}
\item
  Evaluación de las variables asociadas a los factores de riesgo.
\item
  Modelo explicativo con reporte.
\item
  Posible propuesta de indicadores futuros.
\end{itemize}

\end{frame}

\begin{frame}{Fuentes de datos}

\begin{itemize}
\itemsep1pt\parskip0pt\parsep0pt
\item
  CONEVAL: Resago social (censo 2010).
\item
  INEGI:

  \begin{itemize}
  \itemsep1pt\parskip0pt\parsep0pt
  \item
    Censo
  \item
    Encuesta Nacional sobre la Dinaámica de las Relaciones de los
    Hogares (ENDIREH)
  \item
    Encuesta ENVIPE (2013)
  \item
    DENUE
  \end{itemize}
\item
  SEP

  \begin{itemize}
  \itemsep1pt\parskip0pt\parsep0pt
  \item
    Censo educativo (2013).
  \item
    ENLACE (2013).
  \end{itemize}
\item
  ENCUP (Gob e INEGI).
\item
  SINAIS

  \begin{itemize}
  \itemsep1pt\parskip0pt\parsep0pt
  \item
    Egresos hospitalarios
  \item
    Recursos de salud
  \end{itemize}
\item
  SESNSP (Variable dependiente).
\end{itemize}

\end{frame}

\begin{frame}{Problemas con los datos y modelado.}

\begin{itemize}
\itemsep1pt\parskip0pt\parsep0pt
\item
  Años.

  \begin{itemize}
  \itemsep1pt\parskip0pt\parsep0pt
  \item
    De cada fuente de los datos se toma el último año.
  \end{itemize}
\item
  Medición de los factores de riesgo.
\item
  Encuestas

  \begin{itemize}
  \itemsep1pt\parskip0pt\parsep0pt
  \item
    Son estatales.
  \item
    A todos los municipios.
  \item
    Considerar el muestreso de los municipios (No es trivial).
  \end{itemize}
\item
  Espacios públicos.
\item
  NA's.

  \begin{itemize}
  \itemsep1pt\parskip0pt\parsep0pt
  \item
    Registros admin: 0's.
  \item
    Encuestas: muestreo en todos los mun.
  \end{itemize}
\item
  Enlace:

  \begin{itemize}
  \itemsep1pt\parskip0pt\parsep0pt
  \item
    Hay menos registros públicos que los se reportan.
  \end{itemize}
\end{itemize}

\end{frame}

\begin{frame}{Modelo}

\begin{itemize}
\itemsep1pt\parskip0pt\parsep0pt
\item
  La idea es modelar algún agregado de los delitos por municipio.
\item
  Controlar por el tipo de delito.

  \begin{itemize}
  \itemsep1pt\parskip0pt\parsep0pt
  \item
    Violencia, no violencia.
  \item
    Bancos, empresa, otros.
  \item
    Robo común, homicidios, delitos patrimoniales.
  \item
    Transportistas, sindicatos, particulares, negocios.
  \end{itemize}
\item
  Utilizar como variables explicativas los factores de riesgo.

  \begin{itemize}
  \itemsep1pt\parskip0pt\parsep0pt
  \item
    Individuales, familiares, comunitarios, escolares, sociales.
  \end{itemize}
\end{itemize}

\end{frame}

\section{Estadística Descriptiva
Inicial}\label{estadistica-descriptiva-inicial}

\begin{frame}{Embarazo temprano}

\end{frame}

\begin{frame}{Marginación y exclusión social}

\end{frame}

\begin{frame}{Falta de oportunidades laborales, informalidad y
desocupación}

\end{frame}

\begin{frame}{Espacios públicos insuficientes y deteriorados}

\end{frame}

\begin{frame}{Capital social y participación incipiente}

\end{frame}

\begin{frame}{Deserción escolar}

\end{frame}

\begin{frame}{Consumo y abuso de drogas legales e ilegales}

\end{frame}

\begin{frame}{Ambientes familiares deteriorados y problemáticos}

\end{frame}

\end{document}
